\documentclass[a4paper]{scrartcl}

\usepackage[linenumbers,spacing=1,tt,singlebracket]{conan}
\usepackage{covington}

\begin{document}

\title{ConAn}
\subtitle{Typeset Conversation Analysis}
\author{Clemens Horch}

\maketitle

\section{Examples}

Example \ref{conan1} is a numbered conversation analysis example:

\begin{conan}[Anton]
	\alone{Anton}{Lorem ipsum dolor sit amet, consetetur sadipscing elitr, }
	\alone{}{sed diam nonumy eirmod tempor invidunt ut labore et dolore }
	\simul{}{magna aliquyam erat, [ sed diam ] voluptua.}{Zita}{[ ero eos et ] }
	\label{conan1}
\end{conan}


\texttt{conan} environments can easily be mixed with \texttt{example} environments from the \texttt{covington} package. Both use the equation counter for numbering.

\begin{example}
Covington Example
\label{covington1}
\end{example}

The next example is a \texttt{conan*} environment without numbering and custom line numbers.

\setcounter{conanline}{23}
\begin{conan*}[Anton]
	\alone{Anton}{Lorem ipsum dolor sit amet, consetetur sadipscing elitr, }
	\alone{}{sed diam nonumy eirmod tempor invidunt ut labore et dolore }
	\simul{}{magnaaliquyam erat, [ sed diam ] voluptua.}{Zita}{[ ero eos et ] }
\end{conan*}

It is recommended to use ConAn with the conan.py preprocessor. (\ref{example1.conan}) is an example for an automatically processed file.

\input{example1.conan}

\end{document}